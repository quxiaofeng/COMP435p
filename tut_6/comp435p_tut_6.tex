\documentclass[
        %handout,
        %draft,
        ]{beamer}
\usepackage{amssymb,latexsym,amssymb,amsmath,amsbsy,amsopn,amstext,upgreek}
\usepackage{color,multicol}
\usepackage{graphicx,wrapfig,fancybox,watermark,graphics}
\usepackage{picins}
%\usepackage{emp}
%\usepackage{pstricks,pst-plot}
\usepackage{pgf}
\usepackage{movie15}
\usepackage{hyperref}
\usepackage{pdfpages}
\usepackage{listings,bera}
\definecolor{keywords}{RGB}{255,0,90}
\definecolor{comments}{RGB}{60,179,113}
\lstset{language=C,
keywordstyle=\color{keywords},
commentstyle=\color{comments}\emph}
\hypersetup{
    pdfpagemode=FullScreen, % show in full screen?
}
\usepackage{algorithm}
\usepackage{algorithmic}
\renewcommand{\algorithmicrequire}{\textbf{Input:}}
\renewcommand{\algorithmicensure}{\textbf{Output:}}
% reference entry
\usepackage{bibentry, natbib}
% reference style
\bibliographystyle{IEEEtran} 
%reference lib
\nobibliography{refs}

\usepackage[
	compress,
	%minimal,
	%nonav,
	red,
	%gold,
	%numbers,
	%nologo,
	polyu,
	]{beamerthemeHongKong}
\usefonttheme[professionalfonts]{serif}

\title{Tutorial 6}
\author[COMP435p]{Qu Xiaofeng\texorpdfstring{, Teaching Asistant\\\tiny{quxiaofeng.at.polyu@gmail.com}}{}}
\institute{COMP435p\\Biometrics Authentication}
\date{February 27, 2012}



\begin{document}

% include pdf page
\newcommand{\inpdfu}[2]{\begin{figure}\centering\includegraphics[trim=1.4in 6.5in 1.4in 1.4in, clip, height=0.7\textheight, page={#2}]{im/lecture_#1}\end{figure}}
\newcommand{\inpdfl}[2]{\begin{figure}\centering\includegraphics[trim=1.4in 1.4in 1.4in 6.5in, clip, height=0.7\textheight, page={#2}]{im/lecture_#1}\end{figure}}
\newcommand{\inpdfc}[2]{\begin{figure}\centering\includegraphics[trim=0.95in 0.5in 0.95in 0.5in, clip, height=0.7\textheight, page={#2}]{im/lecture_#1}\end{figure}}
% include png images
\newcommand{\inpng}[1]{\begin{figure}\centering\includegraphics[height=0.7\textheight]{im/#1}\end{figure}}

\frame{\titlepage}

\section*{Table of Contents}

    \begin{frame}{\secname}
        \tableofcontents
    \end{frame}

\AtBeginSubsection[] {
    \begin{frame}<handout:0>{Outline}
        \tableofcontents[current,currentsubsection]
    \end{frame}
}



\section{Problems}

    \subsection{Problem 1: Answer The Questions}
    
    		\begin{frame}[t]{Problem 1.1 Two Stages}
    			\begin{overprint}
    				\onslide<1>
    			\emph{Understand two stages (Enrollment and Authentication) in a fingerprint system. (P6:11)}
    				\onslide<2> \inpdfc{7}{11} %P7-11
    				\onslide<3> 
    				\begin{itemize}
    				\item Enrollment stage  				
    					\begin{itemize}
    					\item Image acquisition
    					\item Fingerprint extraction
    					\item Storing
    					\end{itemize}		
    				\item Authentication stage 				
    					\begin{itemize}
    					\item Capture
    					\item Extraction
    					\item Matching
    					\end{itemize}		
    				\end{itemize}							\qed		
    			\end{overprint}
    		\end{frame}
    
    		\begin{frame}[t]{Problem 1.2 Pre-processing}
    			\begin{overprint}
    				\onslide<1>
    			\emph{In the pre-processing stage, what should be done?  (P7:14-16)}
    				\onslide<2> \inpdfc{7}{14} %P7-14
    				\onslide<3> \inpdfc{7}{15} %P7-15
    				\onslide<4> \inpdfc{7}{16} %P7-16
    				\onslide<5> 
    				\begin{itemize}
    				\item Enhancement    				
    					\begin{itemize}
    					\item Noise removal
    					\item Enhance quality of images
    					\item Low-pass filter, FFT-based technique, oriented filters
    					\end{itemize}		
    				\item Segmentation    				
    					\begin{itemize}
    					\item Foreground
    					\item Background
    					\end{itemize}		
    				\item Binarization \& Thinning    				
    					\begin{itemize}
    					\item Binarization by a thresholding operation
    					\item Thinning to reduce the width of ridge to a single pixel
    					\end{itemize}	
    				\end{itemize}							\qed		
    			\end{overprint}
    		\end{frame}
    
    		\begin{frame}[t]{Problem 1.3 Fingerprint sensing}
    			\begin{overprint}
    				\onslide<1>
    			\emph{Understand the two primary methods for fingerprint sensing: inked and live scan and compare the different live-scan fingerprint technologies. (P7:19-25)}
    				\onslide<2> \inpdfc{7}{19} %P7-19
    				(a) belongs to inked, others belong to live-scan
    				\onslide<3> \inpdfc{7}{20} %P7-20
    				Three main kinds.
    				\onslide<4> \inpdfc{7}{21} %P7-21
    				\onslide<5> \inpdfc{7}{22} %P7-22
    				\onslide<6> \inpdfc{7}{23} %P7-23
    				\onslide<7> \inpdfc{7}{24} %P7-24
    				\onslide<8> \inpdfc{7}{25} %P7-25
							\qed		
    			\end{overprint}
    		\end{frame}
    
    		\begin{frame}[t]{Problem 1.4 Global and Local Features}
    			\begin{overprint}
    				\onslide<1>
    			\emph{Please understand the feature definitions in both P7:27-29 (Global feature) and P7: 34-35 (Local feature).  }
    				\onslide<2> \inpdfc{7}{27} %P7-27
    				Global features.
    				\onslide<3> \inpdfc{7}{28} %P7-28
    				\onslide<4> \inpdfc{7}{29} %P7-29
    				\onslide<5> \inpdfc{7}{34} %P7-34
    				\onslide<6> \inpdfc{7}{35} %P7-35
    				Local features.
    				\onslide<7> 
    				\begin{itemize}
    				\item Global features  				
    					\begin{itemize}
    					\item Pattern Area
    					\item Core Point
    					\item Type Lines
    					\item Delta
    					\item Ridge Count
    					\item Basic Ridge Patterns: Loop, Arch, Whorl
    					\end{itemize}		
    				\item Local features  				
    					\begin{itemize}
    					\item Ridge ending
    					\item Ridge bifurcation
    					\item Ridge divergence
    					\item Dot or island
    					\item Enclosure
    					\item Short ridge
    					\end{itemize}	
    				\end{itemize}							\qed		
    			\end{overprint}
    		\end{frame}
    
    		\begin{frame}[t]{Problem 1.5 Matching Strategies}
    			\begin{overprint}
    			\onslide<1>
    			\emph{How many strategies can be used for fingerprint matching? Please compare their difference. (P7:49-52)}
    				\onslide<2> \inpdfc{7}{49} %P7-49
    				Three kinds of matching strategies based on three different represenation of the fingerprint.
    				\onslide<3> \inpdfc{7}{50} %P7-50
    				\onslide<4> \inpdfc{7}{51} %P7-51
    				a Simple way to compute the matching score
    				\onslide<5> \inpdfc{7}{52} %P7-52
    				a General way to compute the matching score
							\qed		
    			\end{overprint}
    		\end{frame}

    \subsection{Problem 2: Singular Points}
    
    		\begin{frame}[t]{\subsecname}
    			\begin{overprint}
    			\onslide<1>
    			\emph{Singular points consist of Core point and Delta point. In P7:30 we can use the directional map to find them (one Core point and two Delta points). Please think about how to make an algorithm to do so. }
    				\onslide<2> \inpdfc{7}{30} %P7-30
    				\onslide<3> 
    				\begin{enumerate}
    				\item Find the most compact circle of the point
    				\item Turn the direction value around the circle into direction vetors
    				\item sum up all the direciton vectors\footnote{Kalle Karu, Anil K. Jain, Fingerprint classification, Pattern Recognition, Volume 29, Issue 3, March 1996, Pages 389-404, ISSN 0031-3203, 10.1016/0031-3203(95)00106-9.}
\footnote{Thien Hoang Van; Hoang Thai Le; , "An Efficient Algorithm for Fingerprint Reference-Point Detection," Computing and Communication Technologies, 2009. RIVF '09. International Conference on , vol., no., pp.1-7, 13-17 July 2009}
    				\item The sum of core points and delta points should be almost zero.
    				\end{enumerate}.
    			\end{overprint}
    		\end{frame}

    \subsection{Problem 3: Fingerprint Patterns}
    
    		\begin{frame}[t]{\subsecname}
    			\begin{overprint}
    			\onslide<1>
    			\emph{In P7:33 six major classes of the overall fingerprint patterns are given. Please compare them and try to classify the other all fingerprint samples in P7:3, 8-9, 12, 14-19, 24, 27, 31.}
    				\onslide<2> \inpdfc{7}{33} %P7-33
    				Arch, Tented arch, Left loop, Right loop, Whorl, Whorl(Twin loop)
    				\onslide<3> \inpdfc{7}{3} %P7-3
    				\alert{Whorl}
    				\onslide<4> \inpdfc{7}{8} %P7-8
    				\alert{Left loop, Whorl, Whorl, Whorl}
    				\onslide<5> \inpdfc{7}{9} %P7-9
    				\alert{Whorl, Arch, Whorl, Right loop}
    				\onslide<6> \inpdfc{7}{12} %P7-12
    				\alert{Whorl, Tented arch}
    				\onslide<7> \inpdfc{7}{14} %P7-14
    				\alert{Left loop}
    				\onslide<8> \inpdfc{7}{15} %P7-15
    				\alert{Arch}
    				\onslide<9> \inpdfc{7}{16} %P7-16
    				\alert{Whorl}    				
    				\onslide<10> \inpdfc{7}{17} %P7-17
    				\alert{Whorl}
    				\onslide<11> \inpdfc{7}{18} %P7-18
    				\onslide<12> \inpdfc{7}{19} %P7-19
    				\alert{Left loop, Whorl, Right loop, Whorl, Whorl(Twin loop)}
    				\onslide<13> \inpdfc{7}{24} %P7-24
    				\alert{Whorl, Whorl}
    				\onslide<14> \inpdfc{7}{27} %P7-27
    				\alert{Whorl}
    				\onslide<15> \inpdfc{7}{31} %P7-31
    				\alert{Arch, Left loop, Whorl}
							\qed		
    			\end{overprint}
    		\end{frame}

    \subsection{Problem 4: Global and Local Fingerprint Features}
    
    		\begin{frame}[t]{\subsecname}
    			\begin{overprint}
    			\onslide<1>
    			\emph{Which features are used as global and local features, respectively? Could you find them in the given fingerprint image (P7:35)?}
    				\onslide<2>
    				\begin{itemize}
    				\item Global features  				
    					\begin{itemize}
    					\item Pattern Area
    					\item Core Point
    					\item Type Lines
    					\item Delta
    					\item Ridge Count
    					\item Basic Ridge Patterns: Loop, Arch, Whorl
    					\end{itemize}		
    				\item Local features  				
    					\begin{itemize}
    					\item Ridge ending
    					\item Ridge bifurcation
    					\item Ridge divergence
    					\item Dot or island
    					\item Enclosure
    					\item Short ridge
    					\end{itemize}	
    				\end{itemize}							
    				\onslide<3> \inpdfc{7}{35} %P7-35
    				
    				\alert{(Left Loop, Delta point: 1\; Bifurcation: 4 and Ending: 3)}							\qed		
    			\end{overprint}
    		\end{frame}

    \subsection{Problem 5: Minutiae Matching}
    
    		\begin{frame}[t]{\subsecname}
    			\begin{overprint}
    			\onslide<1>
    			\emph{Please check the Minutiae matching in P7:46 and compute the matching score according to the Eq.(1) in P7:51.}
    			\onslide<2> \inpdfc{7}{51} %P7-51
    				\onslide<3> \inpdfc{7}{46} %P7-46
    				$S=\frac{100 M_{PQ} M_{PQ}}{M_P M_Q}=\frac{100\times 8 \times 8}{14\times 17}=\alert{26.9}$							\qed		
    			\end{overprint}
    		\end{frame}
    


    \begin{frame}<handout:0>[c]{Q \& A}
        \centerline{\Large{Any questions?}}
    \end{frame}
    
    
    
\end{document}



