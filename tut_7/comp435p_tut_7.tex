\documentclass[
        %handout,
        %draft,
        ]{beamer}
\usepackage{amssymb,latexsym,amssymb,amsmath,amsbsy,amsopn,amstext,upgreek}
\usepackage{color,multicol}
\usepackage{graphicx,wrapfig,fancybox,watermark,graphics}
\usepackage{picins}
%\usepackage{emp}
%\usepackage{pstricks,pst-plot}
\usepackage{pgf}
\usepackage{movie15}
\usepackage{hyperref}
\usepackage{pdfpages}
\usepackage{listings,bera}
\definecolor{keywords}{RGB}{255,0,90}
\definecolor{comments}{RGB}{60,179,113}
\lstset{language=C,
keywordstyle=\color{keywords},
commentstyle=\color{comments}\emph}
\hypersetup{
    pdfpagemode=FullScreen, % show in full screen?
}
\usepackage{algorithm}
\usepackage{algorithmic}
\renewcommand{\algorithmicrequire}{\textbf{Input:}}
\renewcommand{\algorithmicensure}{\textbf{Output:}}
% reference entry
\usepackage{bibentry, natbib}
% reference style
\bibliographystyle{IEEEtran} 
%reference lib
\nobibliography{refs}

\usepackage[
	compress,
	%minimal,
	%nonav,
	red,
	%gold,
	%numbers,
	%nologo,
	polyu,
	]{beamerthemeHongKong}
\usefonttheme[professionalfonts]{serif}

\title{Tutorial 7}
\author[COMP435p]{Qu Xiaofeng\texorpdfstring{, Teaching Asistant\\\tiny{quxiaofeng.at.polyu@gmail.com}}{}}
\institute{COMP435p\\Biometrics Authentication}
\date{March 5, 2012}



\begin{document}

% include pdf page
\newcommand{\inpdfu}[2]{\begin{figure}\centering\includegraphics[trim=1.4in 6.5in 1.4in 1.4in, clip, height=0.7\textheight, page={#2}]{im/lecture_#1}\end{figure}}
\newcommand{\inpdfl}[2]{\begin{figure}\centering\includegraphics[trim=1.4in 1.4in 1.4in 6.5in, clip, height=0.7\textheight, page={#2}]{im/lecture_#1}\end{figure}}
\newcommand{\inpdfc}[2]{\begin{figure}\centering\includegraphics[trim=0.4in 0.1in 0.4in 0.1in, clip, height=0.7\textheight, page={#2}]{im/lecture_#1}\end{figure}}
% include png images
\newcommand{\inpng}[1]{\begin{figure}\centering\includegraphics[height=0.6\textheight]{im/#1}\end{figure}}

\frame{\titlepage}

\section*{Table of Contents}

    \begin{frame}{\secname}
        \tableofcontents
    \end{frame}

\AtBeginSubsection[] {
    \begin{frame}<handout:0>{Outline}
        \tableofcontents[current,currentsubsection]
    \end{frame}
}



\section{Problems}

    \subsection{Problem 1: Answer The Questions}
    
    		\begin{frame}[t]{Problem 1.1 Face recognition system}    		
    			\begin{overprint}
    			\onslide<1>
    			\emph{Understand why use face (P8:3-4) and face recognition system (P8:5-10).}
    			\onslide<2> \inpdfc{8}{3}
    			\onslide<3> \inpdfc{8}{4}
    			\onslide<4> \inpdfc{8}{5}
    			\onslide<5> \inpdfc{8}{6}
    			\onslide<6> \inpdfc{8}{7}
    			\onslide<7> \inpdfc{8}{8}
    			\onslide<8> \inpdfc{8}{9}
    			\onslide<9> \inpdfc{8}{10}
    			\onslide<10>
    				\begin{itemize}
    				\item Face 				
    					\begin{itemize}
    					\item Unique: every one has a unique face
    					\item Passive: can be captured without user cooperation
    					\item Non-intrusive: nature, do not restrict user, more acceptable, human beings tradition
    					\item Less expensive: hardware is cheaper, many alailable database, easy to construct
    					\item Fight terrorism: Increasing need, Requeire automated face detection
    					\item Ubiquitous capture device
    					\item Remote and unobtrusive capture - surveillance
    					\item Much simpler than other ways of biometric personal identification such as fingerprint and iris.
    					\item Cooperation or special knowledge of participants are not required.
    					\end{itemize}		
    				\end{itemize}
    			\onslide<11>
    				\begin{itemize}
    				\item Face recognition system				
    					\begin{itemize}
    					\item Face recognition is the science of programming a
computer to recognize a human face.
    					\item It analyzes facial characteristics
    					\item It requires a digital (web) camera (of low quality is enough).
    					\item This technique has attracted considerable interest.
    					\item Uses distinctive features of the human face in order to verify or identify individuals
    					\item Accuracy: \ldots
							\item Images: \ldots
							\item Capturing images: \ldots
							\item Facial Features: \ldots
							\item Face detection/location: \ldots
							\item Face verification: \ldots
							\item Face identification: \ldots
    					\end{itemize}		
    				\end{itemize}			
    			
    			\qed		
    			\end{overprint}
    		\end{frame}
    
    		\begin{frame}[t]{Problem 1.2 Problems of face recognition}    		
    			\begin{overprint}
    			\onslide<1>
    			\emph{Please point out some problems of face recognition. Can you give their solutions? (P8:11-16)}
    			\onslide<2> \inpdfc{8}{11}
    			Above are the typical problems
    			\onslide<3> \inpdfc{8}{12}
    			Data flow sequence.
    			\onslide<4> \inpdfc{8}{13}
    			Intra-class problems.
    			\onslide<5> \inpdfc{8}{14}
    			Performances
    			\onslide<6> \inpdfc{8}{15}
    			Illumination
    			\onslide<7> \inpdfc{8}{16}
    			Pose\qed		
    			\end{overprint}
    		\end{frame}
    
    		\begin{frame}[t]{Problem 1.3 Three steps of face recognition}    		
    			\begin{overprint}
    			\onslide<1>
    			\emph{What are the general 3 steps of face recognition? (P8:17)}
    			\onslide<2> \inpdfc{8}{17} \qed		
    			\end{overprint}
    		\end{frame}
    
    		\begin{frame}[t]{Problem 1.4 Face detection and location}    		
    			\begin{overprint}
    			\onslide<1>
    			\emph{There are two kinds of the methods in face detection and location, which are statistics-based and knowledge-based method. What difference between them? (P8:18-22)}
    			\onslide<2> \inpdfc{8}{18}
    			\onslide<3> \inpdfc{8}{19}
    			\onslide<4> \inpdfc{8}{20}
    			\onslide<5> \inpdfc{8}{21}
    			\onslide<6> \inpdfc{8}{22}
    			\onslide<7>    			
    				\begin{itemize}
    					\item Statistics-based method
    					\begin{enumerate}
    						\item Subspace method
    						\item NN method\\
    						\alert{The information is learned by statistics data. The learning process can be implemented by programs, which is an automatical process. And the performance can be enhanced by using a larger database.}
							\end{enumerate}
							\item Knowledge-based method
    					\begin{enumerate}
								\item Distribution ruler of gray-value-based
								\item Contour ruler
								\item Color information
								\item Movement Information
								\item Symmetry Information	\\
    						\alert{The grammar is provided by human experts. The performance of these methods greately depends on the knowledge of human experts.}
							\end{enumerate}
    				\end{itemize}		
    			\qed		
    			\end{overprint}
    		\end{frame}
    
    		\begin{frame}[t]{Problem 1.5 Face faeature extraction}    		
    			\begin{overprint}
    			\onslide<1>
    			\emph{There are five methods of face feature extraction (P8:24-41). Please give their basic ideas.}
    			\onslide<2> \inpdfc{8}{24}
    			PCA, Geometry Features, Deformation Models, Neural Networks, Identify Surface
    			\onslide<3> \inpdfc{8}{25}
    			\onslide<4> \inpdfc{8}{26}
    			\onslide<5> \inpdfc{8}{27}
    			\onslide<6> \inpdfc{8}{28}
    			\onslide<7> \inpdfc{8}{29}
    			\onslide<8> \inpdfc{8}{30}
    			\onslide<9> \inpdfc{8}{31}
    			\onslide<10> \inpdfc{8}{32}
    			\onslide<11> \inpdfc{8}{33}
    			\onslide<12> \inpdfc{8}{34}
    			\onslide<13> \inpdfc{8}{35}
    			\onslide<14> \inpdfc{8}{36}
    			\onslide<15> \inpdfc{8}{37}
    			\onslide<16> \inpdfc{8}{38}
    			\onslide<17> \inpdfc{8}{39}
    			\onslide<18> \inpdfc{8}{40}
    			\onslide<19> \inpdfc{8}{41}\qed		
    			\end{overprint}
    		\end{frame}
    
    		\begin{frame}[t]{Problem 1.6 Advantages and disadvantages}    		
    			\begin{overprint}
    			\onslide<1>
    			\emph{What are the advantages and disadvantages of face recognition? (P8:48)}
    			\onslide<2> \inpdfc{8}{48}\qed		
    			\end{overprint}
    		\end{frame}
    		
    \subsection{Problem 2: Eigenfaces algotithm}
    
    		\begin{frame}[t]{\subsecname}
    			\begin{overprint}
    			\onslide<1>
    			\emph{Understand the eigenfaces algorithm (P8:26-29) and eigenfaces recognition algorithm (P8:30-31).}
    				\onslide<2> \inpdfc{8}{26}
    				\onslide<3> \inpdfc{8}{27}
    				\onslide<4> \inpdfc{8}{28}
    				\onslide<5> \inpdfc{8}{29}
    				\onslide<6> \inpdfc{8}{30}
    				\onslide<7> \inpdfc{8}{31} 
    				\onslide<8>  
    				\begin{itemize}
    					\item Eigenfaces algorithm   				
    					\begin{enumerate}
								\item Initialization: Acquire and align an initial set of face images
								\item Compute: the average face image, the difference image for each image in the training set, the covariance matrix of this set of difference images, the eigenvectors of the covariance matrix
								\item Calculate: the corresponding distribution in kdimensional weight space for each known individual by
projecting their face images onto the ``face space.''
							\end{enumerate}
    				\end{itemize}
    				\onslide<9>  
    				\begin{itemize}
							\item Eigenfaces recognition algorithm			
    					\begin{enumerate}
								\item Calculate: a set of weights based on the input image and the M eigenfaces by projecting the input image onto each of the
eigenfaces.
								\item Determine: if the image is a face at all by checking to see if the image is sufficiently close to ``face space.''
								\item Classify: If it is a face, classify the weight pattern as either a known person or as unknown.
							\end{enumerate}
    				\end{itemize}\qed
    			\end{overprint}
    		\end{frame}
    		
    \subsection{Problem 3: Evaluation of the eigenfaces algorithm}
    
    		\begin{frame}[t]{\subsecname}
    			\begin{overprint}
    			\onslide<1>
    			\emph{Eigenface is PCA-based method which is introduced in Lecture 5. At the fifth step of PCA method (P5:23-24) the k most principal components are selected based on the ratio ($\gamma_{\lambda}$) of the eigenvalue sum of selected components to the total sum. Please decide the value of k when the threshold of $\gamma_{\lambda}$ is 95\%. In P8:31 there is a training set including 27 facial images to get eigenfaces. Can you give the value of k according to the training result (eigenfaces)?}
    				\onslide<2> \inpdfu{5}{12} %P5-23
    				\onslide<3> \inpdfl{5}{12} %P5-24
    				The $\gamma_{\lambda}$ of the first seven $\lambda$ is over 95\%.
    				\onslide<4> \inpdfc{8}{31} %P8-31
    				\small{The first nine principal components are of the most information.}\qed
    			\end{overprint}
    		\end{frame}
    		
    \subsection{Problem 4: Eigenfaces}
    
    		\begin{frame}[t]{\subsecname}
    			\begin{overprint}
    			\onslide<1>
    			\emph{P8:34 shows some examples of local features. Please try to define the geometry features based on the woman's face image (refer to P8:34) and give the number of features.}
    				\onslide<2> \inpdfc{8}{34} 
    				\small{The propotions of different length of line segments can be defined as features. And twenty one geometrical features are used.}\qed
    			\end{overprint}
    		\end{frame}   


    \begin{frame}<handout:0>[c]{Q \& A}
        \centerline{\Large{Any questions?}}
    \end{frame}
    
    
    
\end{document}



