\documentclass[
        %handout,
        %draft,
        ]{beamer}
\usepackage{amssymb,latexsym,amssymb,amsmath,amsbsy,amsopn,amstext,upgreek}
\usepackage{color,multicol}
\usepackage{graphicx,wrapfig,fancybox,watermark,graphics}
\usepackage{picins}
%\usepackage{emp}
%\usepackage{pstricks,pst-plot}
\usepackage{pgf}
\usepackage{movie15}
\usepackage{hyperref}
\usepackage{pdfpages}
\usepackage{listings,bera}
\definecolor{keywords}{RGB}{255,0,90}
\definecolor{comments}{RGB}{60,179,113}
\lstset{language=C,
keywordstyle=\color{keywords},
commentstyle=\color{comments}\emph}
\hypersetup{
    pdfpagemode=FullScreen, % show in full screen?
}
\usepackage{algorithm}
\usepackage{algorithmic}
\renewcommand{\algorithmicrequire}{\textbf{Input:}}
\renewcommand{\algorithmicensure}{\textbf{Output:}}
% reference entry
\usepackage{bibentry, natbib}
% reference style
\bibliographystyle{IEEEtran} 
%reference lib
\nobibliography{refs}

\usepackage[
	compress,
	%minimal,
	%nonav,
	red,
	%gold,
	%numbers,
	%nologo,
	polyu,
	]{beamerthemeHongKong}
\usefonttheme[professionalfonts]{serif}

\title{Tutorial 1}
\author[COMP435p]{Qu Xiaofeng\texorpdfstring{, Teaching Asistant\\\tiny{quxiaofeng.at.polyu@gmail.com}}{}}
\institute{COMP435p\\Biometrics Authentication}
\date{January 16, 2012}



\begin{document}

% include pdf page
\newcommand{\inupdf}[2]{\begin{figure}\centering\includegraphics[trim=1.6in 6.5in 1.6in 1.4in, clip, height=0.7\textheight, page={#2}]{im/lecture_#1}\end{figure}}
\newcommand{\inlpdf}[2]{\begin{figure}\centering\includegraphics[trim=1.6in 1.4in 1.6in 6.5in, clip, height=0.7\textheight, page={#2}]{im/lecture_#1}\end{figure}}
% include png images
\newcommand{\inpng}[1]{\begin{figure}\centering\includegraphics[height=0.7\textheight]{im/#1}\end{figure}}

\frame{\titlepage}

\section*{Table of Contents}

    \begin{frame}{\secname}
        \tableofcontents
    \end{frame}

\AtBeginSubsection[] {
    \begin{frame}<handout:0>{Outline}
        \tableofcontents[current,currentsubsection]
    \end{frame}
}




\section{Problems}
    \subsection{Problem 1: Answer the following questions}
        \begin{frame}[c]{Problem 1.a}    		
    				\begin{overprint}
    				\onslide<1>
            		\emph{Understand the reasons why biometrics security in e-world.}
            \onslide<2>%p1-4
                \inlpdf{1}{2}
            \onslide<3>%p1-5
                \inupdf{1}{3}
            \onslide<4>%p1-6
                \inlpdf{1}{3}
            \onslide<5>%p1-7
                \inupdf{1}{4}
            \onslide<6>%p1-8
                \inlpdf{1}{4}
            \onslide<7>%p1-9
                \inupdf{1}{5}
            \onslide<8>%p1-10
                \inlpdf{1}{5}
            \onslide<9>%p1-11
                \inupdf{1}{6}\qed
            \end{overprint}            
        \end{frame}
        
        \begin{frame}[c]{Problem 1.b}
    				\begin{overprint}
    				\onslide<1>
            	\emph{What is biometrics?}
            \onslide<2>%p1-13
                \inupdf{1}{7}
            \onslide<3>%p1-14
                \inlpdf{1}{7}
            \onslide<4>%p2-3
                \inupdf{2}{2}
            \onslide<5>%p2-4
                \inlpdf{2}{2}
            \onslide<6>%p2-5
                \inupdf{2}{3}
            \onslide<7>%p2-6
                \inlpdf{2}{3}\qed
            \end{overprint}            
        \end{frame}
        
        
        
        
        
        \begin{frame}[c]{Problem 1.c}
    				\begin{overprint}
    				\onslide<1>
            	\emph{Understand two types of biometrics and give some examples respectively.}
            \onslide<2>%p2-7
                \inupdf{2}{4}
            \onslide<3>%p2-8
                \inlpdf{2}{4}\qed
            \end{overprint}            
        \end{frame}
        
        
        
        \begin{frame}[c]{Problem 1.d}
    				\begin{overprint}
    				\onslide<1>
            	\emph{The definition of enrollment, template and matching in a biometrics system.}
            \onslide<2>%p2-11
                \inupdf{2}{6}
            \onslide<3>%p2-12
                \inlpdf{2}{6}
            \onslide<4>%p2-13
                \inupdf{2}{7}\qed
            \end{overprint}            
        \end{frame}
        
        
        
        \begin{frame}[c]{Problem 1.e}
    				\begin{overprint}
    				\onslide<1>
            	\emph{What are the requirements of a biometric feature used for authentication purposes? }
            \onslide<2>%p2-20
                \inlpdf{2}{10}
            \onslide<3>%p2-21
                \inupdf{2}{11}
            \onslide<4>%p2-22
                \inlpdf{2}{11}
            \onslide<5>%p2-23
                \inupdf{2}{12}
            \onslide<6>%p2-24
                \inlpdf{2}{12}
            \onslide<7>%p2-25
                \inupdf{2}{13}
            \onslide<8>%p2-26
                \inlpdf{2}{13}
            \onslide<9>%p2-27
                \inupdf{2}{14}
            \onslide<10>%p2-28
                \inlpdf{2}{14}\qed
            \end{overprint}
        \end{frame}
        
        
        
        \begin{frame}[c]{Problem 1.f}
    				\begin{overprint}
    				\onslide<1>
            \emph{Which difference between identification and verification?}
            \onslide<2>%p2-35
                \inupdf{2}{18}
            \onslide<3>%p2-36
                \inlpdf{2}{18}\qed
            \end{overprint}            
        \end{frame}

    \subsection{Problem 2: Biometric authentication system}
        \begin{frame}[c]{\subsecname}
    				\begin{overprint}
    				\onslide<1>
            \emph{What makes up a biometric authentication system? Please give an example to explain them.}
            \onslide<2>%p2-31
                \inupdf{2}{16}
            \onslide<3>%p2-32
                \inlpdf{2}{16}
            \onslide<4>%p2-33
                \inupdf{2}{17}
            \onslide<5>%p2-34
                \inlpdf{2}{17}
            \end{overprint}   
        \end{frame}
        
        
        \begin{frame}[c]{A palmprint biometics system}
            \inpng{p2_example}  
        \end{frame}

    \subsection{Problem 3: Compare}
        \begin{frame}[c]{\subsecname}
    				\begin{overprint}
    				\onslide<1>
            \emph{Compare enrollment mode and authentication/verification mode.}
            \onslide<2>%p2-37
                \inupdf{2}{19}
            \onslide<3>%p2-38
                \inlpdf{2}{19}\qed
            \end{overprint}   
        \end{frame}

    \subsection{Problem 4: FRR, FAR, EER, Crossover, FTE, ATV and ROC}
        \begin{frame}[c]{\subsecname}
            \emph{What are FRR, FAR, EER, Crossover, FTE, ATV and ROC? Now there are two FAR-FRR diagrams and one ROC curve. Give the EER respectively. When FRR=0.1, how about FAR and Crossover, respectively? If the corresponding FTE are all equal to 0.05, then how about ATV? }
        \end{frame}
        
        \begin{frame}[c]{\subsecname terms}
            \begin{overprint}
            \onslide<1>%p2-17
                \inupdf{2}{9}
            \onslide<2>%p2-18
                \inlpdf{2}{9}
            \onslide<3>%p2-19
                \inupdf{2}{10}
            \end{overprint}   
        \end{frame}       
        
        \begin{frame}[c]{\subsecname.a}
            \begin{overprint}
            \onslide<1>
                \inpng{p4_1}
            \onslide<2>
                \begin{align*}
             EER=&0.06\\
             FAR=&0.03\\
             Crossover=&1:round(1/EER)=\frac{1}{17}\\
             ATV=&(1-FTE)(1-FRR)=(1-0.05)\times(1-0.1)=0.855
                \end{align*}
            \end{overprint}   
        \end{frame}       
        
        \begin{frame}[c]{\subsecname.b}
            \begin{overprint}
            \onslide<1>
                \inpng{p4_2}
            \onslide<2>
                \begin{align*}
             EER=&0.09\\
             FAR=&0.07\\
             Crossover=&1:round(1/EER)=\frac{1}{11}\\
             ATV=&(1-FTE)(1-FRR)=(1-0.05)\times(1-0.1)=0.855
                \end{align*}
            \end{overprint}   
        \end{frame}       
        
        \begin{frame}[c]{\subsecname.c}
            \begin{overprint}
            \onslide<1>
                \inpng{p4_3}
            \onslide<2>
                \begin{align*}
             EER=&0.08\\
             FAR=&0.05\\
             Crossover=&1:round(1/EER)=\frac{1}{13}\\
             ATV=&(1-FTE)(1-FRR)=(1-0.05)\times(1-0.1)=0.855
                \end{align*}\qed
            \end{overprint}   
        \end{frame}

    \subsection{Problem 5: Evaluate FAR, FRR and ATV}
        \begin{frame}[c]{\subsecname}
            \emph{100 individuals try to use a biometric system.  There are 38 genuine individuals are accepted, 8 genuine individuals rejected, 44 imposter rejected and 10 imposter accepted. Please evaluate the biometric system by giving the FAR and FRR. If FTE=0.05, then how about ATV?  }
        \end{frame}       
        
        \begin{frame}[c]{\subsecname}
            \begin{overprint}
            \onslide<1>
                \inupdf{2}{8}
            \onslide<2>
                \inlpdf{2}{9}
            \onslide<3>            
            \begin{center}
            \begin{tabular}{lll}
                    \hline%\toprule%
                    Individuals     & Accepted   & Rejected \\\hline%\otoprule%
                    Genuine         & $I_1$      & $I_2$    \\
                    Imposter        & $I_4$      & $I_3$    \\\hline%\bottomrule
            \end{tabular}
            \end{center}
                \begin{align*}
             FAR=&\frac{I_4}{I_3+I_4}\\
             FRR=&\frac{I_2}{I_1+I_2}
                \end{align*}
            \onslide<4>            
            \begin{center}
            \begin{tabular}{lll}
                    \hline%\toprule%
                    Individuals     & Accepted   & Rejected \\\hline%\otoprule%
                    Genuine         & 38         & 8        \\
                    Imposter        & 10         & 44       \\\hline%\bottomrule
            \end{tabular}
            \end{center}
                \begin{align*}
             FAR=&\frac{I_4}{I_3+I_4}=\frac{10}{44+10}=0.19\\
             FRR=&\frac{I_2}{I_1+I_2}=\frac{8}{38+8}=0.17\\
             ATV=&(1-FTE)(1-FRR)=(1-0.05)\times(1-0.17)=0.79
                \end{align*}\qed
            \end{overprint}   
        \end{frame}

    
    \begin{frame}<handout:0>[c]{\secname}
        \centerline{\Large{Any questions?}}
    \end{frame}
    
    
    
\end{document}



