\documentclass[
        %handout,
        %draft,
        ]{beamer}
\usepackage{amssymb,latexsym,amssymb,amsmath,amsbsy,amsopn,amstext,upgreek}
\usepackage{color,multicol}
\usepackage{graphicx,wrapfig,fancybox,watermark,graphics}
\usepackage{picins}
%\usepackage{emp}
%\usepackage{pstricks,pst-plot}
\usepackage{pgf}
\usepackage{movie15}
\usepackage{hyperref}
\usepackage{pdfpages}
\usepackage{listings,bera}
\definecolor{keywords}{RGB}{255,0,90}
\definecolor{comments}{RGB}{60,179,113}
\lstset{language=C,
keywordstyle=\color{keywords},
commentstyle=\color{comments}\emph}
\hypersetup{
    pdfpagemode=FullScreen, % show in full screen?
}
\usepackage{algorithm}
\usepackage{algorithmic}
\renewcommand{\algorithmicrequire}{\textbf{Input:}}
\renewcommand{\algorithmicensure}{\textbf{Output:}}
% reference entry
\usepackage{bibentry, natbib}
% reference style
\bibliographystyle{IEEEtran} 
%reference lib
\nobibliography{refs}

\usepackage[
	compress,
	%minimal,
	nonav,
	red,
	%gold,
	%numbers,
	%nologo,
	polyu,
	]{beamerthemeHongKong}
\usefonttheme[professionalfonts]{serif}

\title{Tutorial 2}
\author[COMP435p]{Qu Xiaofeng\texorpdfstring{, Teaching Asistant\\\tiny{quxiaofeng.at.polyu@gmail.com}}{}}
\institute{COMP435p\\Biometrics Authentication}
\date{January 30, 2012}



\begin{document}

% include pdf page
\newcommand{\inpdfu}[2]{\begin{figure}\centering\includegraphics[trim=1.6in 6.5in 1.6in 1.4in, clip, height=0.7\textheight, page={#2}]{im/lecture_#1}\end{figure}}
\newcommand{\inpdfl}[2]{\begin{figure}\centering\includegraphics[trim=1.6in 1.4in 1.6in 6.5in, clip, height=0.7\textheight, page={#2}]{im/lecture_#1}\end{figure}}
% include png images
\newcommand{\inpng}[1]{\begin{figure}\centering\includegraphics[height=0.7\textheight]{im/#1}\end{figure}}

\frame{\titlepage}

\section*{Table of Contents}

    \begin{frame}{\secname}
        \tableofcontents
    \end{frame}

\AtBeginSubsection[] {
    \begin{frame}<handout:0>{Outline}
        \tableofcontents[current,currentsubsection]
    \end{frame}
}



\section{Problems}
    \subsection{Problem 1: Answer the questions}
        \begin{frame}[c]{Problem 1.a}
    			\begin{overprint}
    			\onslide<1>
            \emph{Understand some useful definitions in image processing: i) pixel, ii) image, iii) image resolution, d) gray-level image.}
            \onslide<2>%p3-6
                \inpdfl{3}{3}
            \onslide<3>%p3-7
                \inpdfu{3}{4}
            \onslide<4>%p3-8
                \inpdfl{3}{4}
            \onslide<5>%p3-9
                \inpdfu{3}{5}
            \onslide<6>%p3-10
                \inpdfl{3}{5}\qed
            \end{overprint}            
        \end{frame}
        
        \begin{frame}[c]{Problem 1.b}
    			\begin{overprint}
    			\onslide<1>
            \emph{What is image histogram? What are peaks and plains in image histogram? Compare the differences between two types of bimodal histograms.}
            \onslide<2>%p3-11
                \inpdfu{3}{6}
            \onslide<3>%p3-12
                \inpdfl{3}{6}\qed
            \end{overprint}            
        \end{frame}
        
        \begin{frame}[c]{Problem 1.c}
    			\begin{overprint}
    			\onslide<1>
            \emph{Understand the point operation. Contrast enhancement is the main application of pixel operation. How many methods of contrast enhancement do you learn? Please compare them.}
            \onslide<2>%p3-14
                \inpdfl{3}{7}
            \onslide<3>%p3-15
                \inpdfu{3}{8}
            \onslide<4>%p3-16
                \inpdfl{3}{8}
            \onslide<5>%p3-17
                \inpdfu{3}{9}
            \onslide<6>%p3-18
                \inpdfl{3}{9}
            \onslide<7>%p3-19
                \inpdfu{3}{10}
            \onslide<8>%p3-20
                \inpdfl{3}{10}
            \onslide<9>%p3-21
                \inpdfu{3}{11}
            \onslide<10>%p3-22
                \inpdfl{3}{11}
            \onslide<11>%p3-23
                \inpdfu{3}{12}\qed
            \end{overprint}            
        \end{frame}
        
        \begin{frame}[c]{Problem 1.d}
    			\begin{overprint}
    			\onslide<1>
            \emph{(d)	Why neighbourhood operations? Give the applications of group operations.}
            \onslide<2>%p3-25
                \inpdfu{3}{13}
            \onslide<3>%p3-26
                \inpdfl{3}{13}\qed
            \end{overprint}            
        \end{frame}
        
    \subsection{Problem 2: Power law function}
        \begin{frame}[c]{\subsecname}
            \emph{What is the power law function? Give the LUT when $\gamma=0.5$ in terms of the table in P3-20. Compare the two LUT and understand the impact of the parameter $\gamma=0.5$.}
        \end{frame}
        
        \begin{frame}[c]{\subsecname terms}
            \begin{overprint}
            \onslide<1>%p3-17
                \inpdfu{3}{9}
            \onslide<2>%p3-18
                \inpdfl{3}{9}
            \onslide<3>%p3-19
                \inpdfu{3}{10}
            \onslide<4>%p3-20
                \inpdfl{3}{10}
            \end{overprint}   
        \end{frame}    
        
        \begin{frame}[c]{\subsecname}
            \begin{overprint}
            \onslide<1>
            	\centering
                \begin{align*}
             I=& & [ & & 0    & & 0.10 & & 0.20 & & 0.30 & & 0.40 & & 0.50 & & \\
               & &   & & 0.60 & & 0.70 & & 0.80 & & 0.90 & & 1.00 & &      & & ]\\
             O=& &   & & I^\gamma\\
             O=& & [ & & 0^{0.5}       & & 0.10^{0.5}    & & 0.20^{0.5}    
                     & & 0.30^{0.5}    & & 0.40^{0.5}    & & 0.50^{0.5}    & & \\
               & &   & & 0.60^{0.5}    & & 0.70^{0.5}    & & 0.80^{0.5} 
                     & & 0.90^{0.5}    & & 1.00^{0.5}    & &               & & ]\\                
              =& & [ & & 0    & & 0.32 & & 0.45 & & 0.55 & & 0.63 & & 0.71 & & \\
               & &   & & 0.77 & & 0.84 & & 0.89 & & 0.95 & & 1.00 & &      & & ]\\
               \qed
                \end{align*}
            \end{overprint}   
        \end{frame}  
        
    \subsection{Problem 3: Convolution}
        \begin{frame}[c]{\subsecname}
            \emph{Please check the convolution result in P3-33 and try to answer the question in P3-34. Now there is an input image and a mask as below. Could you give the convolved image?}
        \end{frame}
        
        \begin{frame}[c]{\subsecname terms}
            \begin{overprint}
            \onslide<1>%p3-27
                \inpdfu{3}{14}
            \onslide<2>%p3-28
                \inpdfl{3}{14}
            \onslide<3>%p3-29
                \inpdfu{3}{15}
            \onslide<4>%p3-30
                \inpdfl{3}{15}
            \onslide<5>%p3-31
                \inpdfu{3}{16}
            \onslide<6>%p3-32
                \inpdfl{3}{16}
            \onslide<7>%p3-33
                \inpdfu{3}{17}
            \onslide<8>%p3-34
                \inpdfl{3}{17}
            \end{overprint}   
        \end{frame}    
        
        \begin{frame}[c]{\subsecname}
            \begin{overprint}
            \onslide<1>
            The convolved image size for a 128$\times$128 image and a 7$\times$7 mask is 122$\times$122.
            \onslide<2>
            \begin{center} 
            Mask\\           
            \begin{tabular}{|r|r|r|}
                     \hline
                    -1 & 2 & 1 \\\hline
                    -1 & 2 & 1 \\\hline
            \end{tabular}\\$\;$\\
            Input image\\           
            \begin{tabular}{|r|r|r|r|r|}
                    \hline
                    4 & 4 & 7  & 1 & 9 \\\hline
                    4 & 3 & 8  & 2 & 9 \\\hline
                    3 & 5 & 9  & 8 & 9 \\\hline
                    3 & 6 & 9  & 8 & 9 \\\hline
                    3 & 6 & 10 & 9 & 9 \\\hline
            \end{tabular}
            \end{center}
            \onslide<3>
            \begin{center} 
            Output image\\           
            \begin{tabular}{|r|r|r|}
                    \hline
                    21 & 26 &  9 \\\hline
                    26 & 37 & 21 \\\hline
                    34 & 41 & 42 \\\hline
                    37 & 43 & 33 \\\hline
            \end{tabular}
            \end{center}\qed
            \end{overprint}
        \end{frame}  
    
    \subsection{Problem 4: Noise reduction}
        \begin{frame}[c]{\subsecname}
            \emph{There are two methods of noise reduction: low pass filter and median filter (see P3: 37-46). Please give the noise reduction results of the input image in Q3 by using the two methods according to the conditions: a) the mask of low pass filter is $ \frac{1}{9} \begin{bmatrix} 1 & 1 & 1\\ 1 & 1 & 1\\ 1 & 1 & 1 \end{bmatrix} $, and b) the mask of median filter is $ \begin{bmatrix} 1 & 1 & 1\\ 1 & 1 & 1\\ 1 & 1 & 1 \end{bmatrix} $ }
        \end{frame}
        
        \begin{frame}[c]{\subsecname terms}
            \begin{overprint}
            \onslide<1>%p3-37
                \inpdfu{3}{19}
            \onslide<2>%p3-38
                \inpdfl{3}{19}
            \onslide<3>%p3-39
                \inpdfu{3}{20}
            \onslide<4>%p3-40
                \inpdfl{3}{20}
            \onslide<5>%p3-41
                \inpdfu{3}{21}
            \onslide<6>%p3-42
                \inpdfl{3}{21}
            \onslide<7>%p3-43
                \inpdfu{3}{22}
            \onslide<8>%p3-44
                \inpdfl{3}{22}
            \onslide<9>%p3-45
                \inpdfu{3}{23}
            \onslide<10>%p3-46
                \inpdfl{3}{23}
            \end{overprint}   
        \end{frame}    
        
        \begin{frame}[c]{\subsecname }
            \begin{overprint}
            \onslide<1>
            \begin{center} 
            Input image\\           
            \begin{tabular}{|r|r|r|r|r|}
                    \hline
                    4 & 4 & 7  & 1 & 9 \\\hline
                    4 & 3 & 8  & 2 & 9 \\\hline
                    3 & 5 & 9  & 8 & 9 \\\hline
                    3 & 6 & 9  & 8 & 9 \\\hline
                    3 & 6 & 10 & 9 & 9 \\\hline
            \end{tabular}
            \end{center}
            $\;$\\            
            \begin{columns}
            \column{0.5\textwidth}
            \begin{center} 
            Low pass filter mask\\           
            $ \frac{1}{9} \begin{bmatrix} 1 & 1 & 1\\ 1 & 1 & 1\\ 1 & 1 & 1 \end{bmatrix} $
            \end{center}
            \column{0.5\textwidth}
            \begin{center}            
            Result a)\\           
            \begin{tabular}{|r|r|r|}
                    \hline
                    5 & 5 & 7  \\\hline
                    6 & 6 & 8  \\\hline
                    6 & 8 & 9  \\\hline
            \end{tabular}
            \end{center}
            \end{columns}
            \onslide<2>
            \begin{center} 
            Input image\\           
            \begin{tabular}{|r|r|r|r|r|}
                    \hline
                    4 & 4 & 7  & 1 & 9 \\\hline
                    4 & 3 & 8  & 2 & 9 \\\hline
                    3 & 5 & 9  & 8 & 9 \\\hline
                    3 & 6 & 9  & 8 & 9 \\\hline
                    3 & 6 & 10 & 9 & 9 \\\hline
            \end{tabular}
            \end{center}
            $\;$\\            
            \begin{columns}
            \column{0.5\textwidth}
            \begin{center} 
            Median filter mask\\           
            $ \begin{bmatrix} 1 & 1 & 1\\ 1 & 1 & 1\\ 1 & 1 & 1 \end{bmatrix} $
            \end{center}
            \column{0.5\textwidth}
            \begin{center} 
            Result b)\\           
            \begin{tabular}{|r|r|r|}
                    \hline
                    4 & 5 & 8  \\\hline
                    5 & 8 & 9  \\\hline
                    6 & 8 & 9  \\\hline
            \end{tabular}
            \end{center}
            \end{columns}
            \qed
            \end{overprint}
        \end{frame}  
    \subsection{Problem 5: Filter operations}    
        \begin{frame}[c]{\subsecname}
            \emph{In P3: 47-49 there are some examples of filter operations. Please point out the difference of Low-pass filters, High-pass filters and Edge enhancement filters. Notice that the `'0''-Sum mask for Edge enhancement filters and `'1''-Sum for others. }
        \end{frame}
        
        \begin{frame}[c]{\subsecname terms}
            \begin{overprint}
            \onslide<1>%p3-47
                \inpdfu{3}{24}
            \onslide<2>%p3-48
                \inpdfl{3}{24}
            \onslide<3>%p3-49
                \inpdfu{3}{25}
            \end{overprint}   
        \end{frame}  
    
    \begin{frame}<handout:0>[c]{\secname}
        \centerline{\Large{Any questions?}}
    \end{frame}
    
    
    
\end{document}



