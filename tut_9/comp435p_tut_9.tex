\documentclass[
        %handout,
        %draft,
        ]{beamer}
\usepackage{amssymb,latexsym,amssymb,amsmath,amsbsy,amsopn,amstext,upgreek}
\usepackage{color,multicol}
\usepackage{graphicx,wrapfig,fancybox,watermark,graphics}
\usepackage{picins}
%\usepackage{emp}
%\usepackage{pstricks,pst-plot}
\usepackage{pgf}
\usepackage{movie15}
\usepackage{hyperref}
\usepackage{pdfpages}
\usepackage{listings,bera}
\definecolor{keywords}{RGB}{255,0,90}
\definecolor{comments}{RGB}{60,179,113}
\lstset{language=C,
keywordstyle=\color{keywords},
commentstyle=\color{comments}\emph}
\hypersetup{
    pdfpagemode=FullScreen, % show in full screen?
}
\usepackage{algorithm}
\usepackage{algorithmic}
\renewcommand{\algorithmicrequire}{\textbf{Input:}}
\renewcommand{\algorithmicensure}{\textbf{Output:}}
% reference entry
\usepackage{bibentry, natbib}
% reference style
\bibliographystyle{IEEEtran} 
%reference lib
\nobibliography{refs}

\usepackage[
	compress,
	%minimal,
	%nonav,
	red,
	%gold,
	%numbers,
	%nologo,
	polyu,
	]{beamerthemeHongKong}
\usefonttheme[professionalfonts]{serif}

\title{Tutorial 9}
\author[COMP435p]{Qu Xiaofeng\texorpdfstring{, Teaching Asistant\\\tiny{quxiaofeng.at.polyu@gmail.com}}{}}
\institute{COMP435p\\Biometrics Authentication}
\date{March 19, 2012}



\begin{document}

% include pdf page
\newcommand{\inpdfu}[2]{\begin{figure}\centering\includegraphics[trim=1.4in 6.5in 1.4in 1.4in, clip, height=0.7\textheight, page={#2}]{im/lecture_#1}\end{figure}}
\newcommand{\inpdfl}[2]{\begin{figure}\centering\includegraphics[trim=1.4in 1.4in 1.4in 6.5in, clip, height=0.7\textheight, page={#2}]{im/lecture_#1}\end{figure}}
\newcommand{\inpdfc}[2]{\begin{figure}\centering\includegraphics[trim=0.15in 0.2in 0.15in 0.2in, clip, height=0.75\textheight, page={#2}]{im/lecture_#1}\end{figure}}
% include png images
\newcommand{\inpng}[1]{\begin{figure}\centering\includegraphics[height=0.55\textheight]{im/#1}\end{figure}}

\frame{\titlepage}

\section*{Table of Contents}

    \begin{frame}{\secname}
        \tableofcontents
    \end{frame}

\AtBeginSubsection[] {
    \begin{frame}<handout:0>{Outline}
        \tableofcontents[current,currentsubsection]
    \end{frame}
}



\section{Problems}

    \subsection{Problem 1: Answer The Questions}
    
    		\begin{frame}[t]{Problem 1.1 Voice biometrics}    		
    			\begin{overprint}
    			\onslide<1>
    				\emph{Understand the difference between voice biometrics and speech recognition? (P11:10). What is text-dependent and text-independent speaker ID? (P11:11).}
    			\onslide<2> \inpdfc{11}{10}
    			\onslide<3> \inpdfc{11}{11}
    			\onslide<4>
    				\begin{itemize}
    				\item Differences
    					\begin{itemize}
    					\item Voice biometrics focuses on the identity of the speaker, different content of key words could be used; speech recognition focuses on the content of the speech, no matter who is speaking.
    					\end{itemize}		
    				\item Text-dependent speaker ID
    					\begin{itemize}
    					\item The utterance of key words or sentences are the same for both training and recognition.
    					\end{itemize}				
    				\item Text-independentspeaker ID
    					\begin{itemize}
    					\item To verify the identities of the people by the way they speak. 
    					\end{itemize}		
    				\end{itemize}\qed		
    			\end{overprint}
    		\end{frame}
    
    		\begin{frame}[t]{Problem 1.2 Signature recognition}    		
    			\begin{overprint}
    			\onslide<1>
    			\emph{Generally speaking, on-line signature recognition is more accurate than off-line. Can you give some reasons? (P10:6-7)}
    			\onslide<2> \inpdfc{10}{6}
    			\onslide<3> \inpdfc{10}{7}
    			\onslide<4>
    				\begin{itemize}
    				\item Advantages of online solution
    					\begin{itemize}
    					\item pressure
    					\item speed, acceleration
    					\end{itemize}		
    				\end{itemize}\qed		
    			\end{overprint}
    		\end{frame}
    
    		\begin{frame}[t]{Problem 1.3 Pros and cons}    		
    			\begin{overprint}
    			\onslide<1>
    			\emph{What are the advantages and disadvantages of voice and signature biometrics, respectively? (P10: 34; P11: 27)}
    			\onslide<2> \inpdfc{10}{34}
    			\onslide<3> \inpdfc{11}{27}\qed	
    			\end{overprint}
    		\end{frame}
    
    		\begin{frame}[t]{Problem 1.4 Two features of signature verification}    		
    			\begin{overprint}
    			\onslide<1>
    			\emph{Show both global and local features for on-line signature verification.  (P10: 23)}
    			\onslide<2> \inpdfc{10}{23}\qed	
    			\end{overprint}
    		\end{frame}
    		
    \subsection{Problem 2: Feature set of voice identification}
    
    		\begin{frame}[t]{\subsecname}
    			\begin{overprint}
    			\onslide<1>
    			\emph{Features used by voice identification include cadence, frequency, pitch \& tone of an individual's voice. In P11:12 the models for two speakers saying the same vowel are given. Please try to define a minimum feature set to divide the two models.}
    			\onslide<2> \inpdfc{11}{12}
    			\alert{(frequency, magnitude)}\qed	
    			\end{overprint}
    		\end{frame}
    		
    \subsection{Problem 3: Directional frontier}
    
    		\begin{frame}[t]{\subsecname}
    			\begin{overprint}
    			\onslide<1>
    			\emph{Directional frontier (DF) is a directional grouping of the contour pixels. In P10:25, there is an example of DF curve characterization. Please give the corresponding feature set.}
    			\onslide<2> \inpdfc{10}{25}
    			\onslide<3>
    				\begin{itemize}
    				\item end points	
    				\item angle (major orientation)
    				\item skew (line of best-fit)
    				\item via points
    				\item spline curve
    				\item middle point
    				\item number of points
    				\end{itemize}\qed	
    			\end{overprint}
    		\end{frame}
    		
    \subsection{Problem 4: Features of signature}
    
    		\begin{frame}[t]{\subsecname}
    			\begin{overprint}
    			\onslide<1>
    			\emph{Please write down your signature. Based on the definitions given in P10:11-14, please find the following features: Upper/lower envelop, vertical/horizontal projection, geometric and topological features. }
    			\onslide<2> \inpdfc{10}{11} 
    			\onslide<3> \inpdfc{10}{12} 
    			\onslide<4> \inpdfc{10}{13} 
    			\onslide<5> \inpdfc{10}{14}\qed	
    			\end{overprint}
    		\end{frame}   
    		
    \begin{frame}<handout:0>[c]{Q \& A}
        \centerline{\Large{Any questions?}}
    \end{frame}
    
    
    
\end{document}



