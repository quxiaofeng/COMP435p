\documentclass[
        %handout,
        %draft,
        ]{beamer}
\usepackage{amssymb,latexsym,amssymb,amsmath,amsbsy,amsopn,amstext,upgreek}
\usepackage{color,multicol}
\usepackage{graphicx,wrapfig,fancybox,watermark,graphics}
\usepackage{picins}
%\usepackage{emp}
%\usepackage{pstricks,pst-plot}
\usepackage{pgf}
\usepackage{movie15}
\usepackage{hyperref}
\usepackage{pdfpages}
\usepackage{listings,bera}
\definecolor{keywords}{RGB}{255,0,90}
\definecolor{comments}{RGB}{60,179,113}
\lstset{language=C,
keywordstyle=\color{keywords},
commentstyle=\color{comments}\emph}
\hypersetup{
    pdfpagemode=FullScreen, % show in full screen?
}
\usepackage{algorithm}
\usepackage{algorithmic}
\renewcommand{\algorithmicrequire}{\textbf{Input:}}
\renewcommand{\algorithmicensure}{\textbf{Output:}}
% reference entry
\usepackage{bibentry, natbib}
% reference style
\bibliographystyle{IEEEtran} 
%reference lib
\nobibliography{refs}

\usepackage[
	compress,
	%minimal,
	%nonav,
	red,
	%gold,
	%numbers,
	%nologo,
	polyu,
	]{beamerthemeHongKong}
\usefonttheme[professionalfonts]{serif}

\title{Tutorial 4}
\author[COMP435p]{Qu Xiaofeng\texorpdfstring{, Teaching Asistant\\\tiny{quxiaofeng.at.polyu@gmail.com}}{}}
\institute{COMP435p\\Biometrics Authentication}
\date{February 13, 2012}



\begin{document}

% include pdf page
\newcommand{\inpdfu}[2]{\begin{figure}\centering\includegraphics[trim=1.4in 6.5in 1.4in 1.4in, clip, height=0.7\textheight, page={#2}]{im/lecture_#1}\end{figure}}
\newcommand{\inpdfl}[2]{\begin{figure}\centering\includegraphics[trim=1.4in 1.4in 1.4in 6.5in, clip, height=0.7\textheight, page={#2}]{im/lecture_#1}\end{figure}}
% include png images
\newcommand{\inpng}[1]{\begin{figure}\centering\includegraphics[height=0.7\textheight]{im/#1}\end{figure}}

\frame{\titlepage}

\section*{Table of Contents}

    \begin{frame}{\secname}
        \tableofcontents
    \end{frame}

\AtBeginSubsection[] {
    \begin{frame}<handout:0>{Outline}
        \tableofcontents[current,currentsubsection]
    \end{frame}
}



\section{Problems}

    \subsection{Problem 1: Answer the questions}
    
    		\begin{frame}[c]{Problem 1.1 scatter matrix, eigenvector and eigenvalue}
    			\begin{overprint}
    				\onslide<1>
    			\emph{What is the scatter matrix (P5:21)? Understand what about eigenvector and eigenvalue as well as their functions? (P5:22)}
    				\onslide<2> \inpdfu{5}{11} %P5-21
    				\onslide<3> \inpdfl{5}{11} %p5-22
    				\onslide<4> $\;$\\
\alert{\href{http://www.google.com.hk/search?gcx=c&sourceid=chrome&ie=UTF-8&q=eigenvalue}{''Detailed links(click)''}}\\	\alert{\href{http://books.google.com.hk/books/about/Principal_component_analysis.html?id=_olByCrhjwIC&redir_esc=y
}{''PCA books(click)''}}\\
							\alert{kolho3.tiera.ru}    		
							\qed		
    			\end{overprint}
    		\end{frame}
    
    		\begin{frame}[c]{Problem 1.2 Eigenface}
    			\begin{overprint}
    				\onslide<1>
    			\emph{Understand the PCA application to facial recognition: Eigenface. (P5:26-30)}
    				\onslide<2> \inpdfl{5}{13} %P5-26
    				\onslide<3> \inpdfu{5}{14} %p5-27
    				\onslide<4> \inpdfl{5}{14} %p5-28
    				\onslide<5> \inpdfu{5}{15} %p5-29
    				\onslide<6> \inpdfl{5}{15} %p5-30
    				\qed
    			\end{overprint}
    		\end{frame}
    
    		\begin{frame}[c]{Problem 1.3 PCA and LDA}
    			\begin{overprint}
    				\onslide<1>
    			\emph{Compare two StatPR techniques, PCA and LDA (P5:13) and point out their main difference (P5:42-45)}
    				\onslide<2> \inpdfu{5}{7} %p5-13    				
    				\onslide<3> \inpdfl{5}{21} %p5-42
    				\onslide<4> \inpdfu{5}{22} %p5-43
    				\onslide<5> \inpdfl{5}{22} %p5-44
    				\onslide<6> \inpdfu{5}{23} %p5-45
    				\qed
    			\end{overprint}
    		\end{frame}
    
    		\begin{frame}[c]{Problem 1.4 LDA}
    			\begin{overprint}
    				\onslide<1>
    			\emph{Linear discrimination analysis (LDA) is introduced in P5:37-40. Please understand the two steps in P5:37 and compare within-class scatter matrix with between-class scatter matrix.}
    				\onslide<2> \inpdfu{5}{19} %p5-37
    				\onslide<3> \inpdfl{5}{19} %p5-38
    				\onslide<4> \inpdfu{5}{20} %p5-39
    				\onslide<5> \inpdfl{5}{20} %p5-40
    				\qed
    			\end{overprint}
    		\end{frame}

    \subsection{Problem 2: Three PR Approaches}
    
    		\begin{frame}[c]{\subsecname}
    			\begin{overprint}
    				\onslide<1>
    			\emph{There are three PR approaches: StatPR, SyatPR and NN. What difference between them (P5:4-10)? Based on your knowledge, can you give a simple application for each approach?}
    				\onslide<2> \inpdfl{5}{2} %p5-4
    				\onslide<3> \inpdfu{5}{3} %p5-5
    				\onslide<4> \inpdfl{5}{3} %p5-6
    				\onslide<5> \inpdfu{5}{4} %p5-7
    				\onslide<6> \inpdfl{5}{4} %p5-8
    				\onslide<7> \inpdfu{5}{5} %p5-9
    				\onslide<8> \inpdfl{5}{5} %p5-10
    				\onslide<9>
    					\begin{itemize}
    						\item
The properties of the patterns (entities) are different: 
								\item
For statistical PR the patterns can be extracted a set of features and there is an underlying statistical model for the generation of theses patterns.
								\item
For syntactic PR the patterns should contain structural or relational information and the structure of an entity is paramount. 
								\item
For NN, the patterns are stored in the Network, and it can't be seen.
  						\end{itemize}
    				\onslide<10> 
    					\begin{itemize}
								\item
The methods are different:
								\item
  For statistical PR the problems are solved by using statistical theory.
								\item
  For syntactic PR the classification can be accomplished by defining suitable and distinct grammars that reflect the structure of each pattern class.
								\item
  For NN, the result is learned by the neurons in the network.
  						\end{itemize}
    				\onslide<11> $\;$\\
    				\begin{center}
            Examples\\           
            \begin{tabular}{|c|c|}
                    \hline
                    Approaches & Examples        \\\hline
                    StatPR     & Face, Palmprint \\
                    SyatPR     & Disease    \\
                    NN         & Motor Control   \\\hline
            \end{tabular}
            \end{center}
            \qed
    			\end{overprint}
    		\end{frame}

    \subsection{Problem 3: StatPR and SyatPR}
    
    		\begin{frame}[c]{\subsecname}
    			\begin{overprint}
    				\onslide<1>
    			\emph{Please check the examples of PR approaches in P5:10. Try to analyze the character ``H'' by statistical PR approach and structural PR approach.}
    				\onslide<2> \inpdfl{5}{5} %p5-10
    				\onslide<3> $\;$\\
    					\begin{block}{StatPR approach}
    						The feature set:(intersections, -, |, holes)\\
    						$ x=[2, 1, 4, 0] $
    					\end{block}
    					\begin{block}{SyntPR approach}
    						Primitives and relations:$\uparrow + \uparrow \alert{\rightrightarrows} \rightarrow \alert{\rightrightarrows} \uparrow + \uparrow $
    					\end{block}
    					\qed
    			\end{overprint}
    		\end{frame}

    \subsection{Problem 4: PCA}
    
    		\begin{frame}[c]{\subsecname}
    			\begin{overprint}
    				\onslide<1>
    			\emph{From P5:19-25, PCA method given by using image data is defined, which projects an image space with 644 dimension into 6 dimension eigenvector space. Please understand each step.}
    				\onslide<2> \inpdfu{5}{10} %p5-19
    				\onslide<3> \inpdfl{5}{10} %p5-20
    				\onslide<4> \inpdfu{5}{11} %p5-21
    				\onslide<5> \inpdfl{5}{11} %p5-22
    				\onslide<6> \inpdfu{5}{12} %p5-23
    				\onslide<7> \inpdfl{5}{12} %p5-24
    				\onslide<8> \inpdfu{5}{13} %p5-25
    				\qed
    			\end{overprint}
    		\end{frame}

    \subsection{Problem 5: Three PR Approaches}
    
    		\begin{frame}[c]{\subsecname}
    			\begin{overprint}
    				\onslide<1>
    			\emph{According to the figure in P5:24, how to find its minimum $\lambda$ if we hope to get $\gamma_{\lambda}>60$?}
    				\onslide<2> \inpdfl{5}{12} %p5-24
    				When $\lambda=2$ the $\gamma_{\lambda}<60$When \alert{$\lambda=3$} the $\gamma_{\lambda}>60$, so $\lambda=3$ is what we want.\qed
    			\end{overprint}
    		\end{frame}

    \begin{frame}<handout:0>[c]{\secname}
        \centerline{\Large{Any questions?}}
    \end{frame}
    
    
    
\end{document}



